
\documentclass[12pt,a4]{article}

\usepackage[utf8]{inputenc}
\usepackage[T1]{fontenc}
\usepackage{amsmath, amssymb, amsfonts}
\usepackage{mathtools}
\usepackage{graphicx, float, epstopdf}
\usepackage{enumerate}
\usepackage{bm}
\usepackage[noabbrev]{cleveref}
\DeclareMathOperator*{\argmin}{\arg\!\min}

\newcommand{\R}{{\mathbb R}}
\newcommand{\C}{{\mathbb C}}
\newcommand{\N}{{\mathbb N}}
\newcommand{\ra}{\rightarrow}
\newcommand{\lra}{\longrightarrow}
\newcommand{\lnorm}{\left\|}
\newcommand{\rnorm}{\right\|}
\newcommand{\ind}{{\mathbf{1}}}

\newcommand{\citeneeded}{\ensuremath{\vphantom{1}^{\text{[citation needed]}}\,}}
\newcommand{\eps}{\ensuremath{\varepsilon}}
\newcommand{\vc}[1]{\ensuremath{\bm{#1}}}
\newcommand*{\bigCI}{%
  \mathrel{\text{%
    {\rotatebox[origin=c]{90}{\resizebox{2.25ex}{1.65ex}{$\vDash$}}}%
  }}%
}

\title{Expected value of a sample of a standard multivariate normal distribution}
\author{Lasse Lybeck, Robert Sirviö}


\begin{document}

\maketitle

\section{Introduction}

In this work we will discuss the expected value of a sample of a standard multivariate normal distribution. More precisely, we will study the claim in \cite{samu}, section 5.4.1 (p. 72), which states, that if the elements of a noise vector $\eps \in \R^k$ satisfy $e_j \sim N(0, \sigma^2)$, then the expected value satisfies $E \left( \lnorm \eps \rnorm \right) = \sqrt{k} \sigma$. We will see that this claim is imprecise. It turns out however, that the statement is \emph{asymptotically} true, and that the convergence toward the precise result is very fast.

\section{Calculations}
\label{sec:calc}

Let
\begin{equation}
\vc{\eps'} = \left(Z_1, \ldots, Z_n \right), \quad Z_i \sim N(0,1) \bigCI \text{ for all }
i \in \left\{1, \ldots, n\right\} 
\end{equation}
and define
\begin{equation}
\label{eq:epsdef}
\vc{\eps} := \sigma \vc{\eps'} = \left(\sigma Z_1, \ldots, \sigma Z_n \right), \quad \sigma > 0.
\end{equation}
Now it holds that
\begin{equation}
\sigma Z_i \sim N\left(0,\sigma^2\right) \text{ for all }
i \in \left\{1, \ldots, n\right\}.
\end{equation}

Let now $X \sim \chi_n^2$ and $Y \sim \chi_n$, where $\chi_n^2$ denotes the chi-squared distribution and $\chi_n$ the chi distribution, both with $n$ degrees of freedom. We now see (\cite{handbook}, \cite{chi}, \cite{mathworld}) that
\begin{equation}
\label{eq:calc}
\begin{alignedat}{1}
E\left(\lnorm \vc{\eps} \rnorm \right)   &= E\left(\lnorm \sigma \vc{\eps'} \rnorm \right)   =
E\left(\sigma \lnorm \vc{\eps'} \rnorm \right) = \sigma E\left(\lnorm \vc{\eps'} \rnorm \right) \\
&= \sigma E\left(\sqrt{Z_1^2 + \ldots + Z_n^2} \right) = \sigma E\left(\sqrt{X} \right) \\
&= \sigma E\left( Y \right) = 
\sigma \sqrt{2}\frac{\Gamma\left(\frac{n+1}{2}\right)}{\Gamma\left(\frac{n}{2}\right)} \\
&= \sigma \sqrt{2} \sqrt{\frac{n}{2}} \left( 1 - \frac{1}{8 n/2} + \frac{1}{128 (n/2)^2} + \frac{5}{1024 (n/2)^3} \right. \\ & \qquad \qquad \qquad \qquad \left. - \frac{21}{32768 (n/2)^4} + O\left( n^{-5} \right) \right) \\
&= \sigma \sqrt{n} \left( 1 - \frac{1}{4 n} + \frac{1}{32 n^2} + \frac{5}{128 n^3} - \frac{21}{2048 n^4} + O\left( n^{-5} \right) \right) .
\end{alignedat}
\end{equation}

\section{Discussion}

As we can see from \cref{eq:calc} in \cref{sec:calc}, it is clear that for every $n > 0$ and $\eps$ as defined in \cref{eq:epsdef} we get
\begin{equation}
E \left( \lnorm \eps \rnorm \right) < \sigma \sqrt{n} .
\end{equation}
However, it is also easy to see that 
\begin{equation}
\lim_{n \rightarrow \infty} E \left( \lnorm \eps \rnorm \right) = \sigma \sqrt{n} .
\end{equation}


%----------------------References---------------------
%\newpage
\begin{thebibliography}{9}

\begin{footnotesize}
    
\bibitem{handbook}
    Abramowitz, M \& Stegun, I.A. (1965)
    \emph{Handbook of mathematical functions with formulas, graphs and mathenatical tables}
    New York, NY: Dover, p. 940
    
\bibitem{chi}
    Evans, M., Hastings, N. \& Peacock, B. (2000)
    \emph{Statistical distributions}.
    New York: Wiley, p. 57
    
\bibitem{samu}
	Mueller, Jennifer L. \& Siltanen Samuli, (2012).
	\emph{Linear and Nonlinear Inverse Problems with Practical Applications}.
	SIAM, 1st edition.

\bibitem{mathworld}
    http://mathworld.wolfram.com/GammaFunction.html. Wolfram Mathworld: Gamma Function, equation (98). Last accessed 9th May 2014.


\end{footnotesize}

\end{thebibliography}

\end{document}



