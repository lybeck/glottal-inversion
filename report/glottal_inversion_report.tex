\documentclass[12pt,a4]{article}

\usepackage[utf8]{inputenc}
\usepackage[T1]{fontenc}
\usepackage{amsmath,amsfonts}
\usepackage{graphicx}
\usepackage{enumerate}
\DeclareMathOperator*{\argmin}{\arg\!\min}

\newcommand{\R}{{\mathbb R}}
\newcommand{\C}{{\mathbb C}}
\newcommand{\N}{{\mathbb N}}
\newcommand{\ra}{\rightarrow}
\newcommand{\lra}{\longrightarrow}
\newcommand{\lnorm}{\left\|}
\newcommand{\rnorm}{\right\|}
\newcommand{\ind}{{\mathbf{1}}}


\title{Glottal Inversion}
\author{Lasse Lybeck\\Robert Sirviö}


\begin{document}

\maketitle

%---------------------Introduction--------------------
\section{Introduction}\label{sec:intro}


%-----------------Materials & Methods-----------------
\section{Materials and Methods}\label{sec:methods}

\subsection{The inversion method}\label{subsec:invMethod}

\subsubsection{Tikhonov reguralization}\label{subsubsec:tikh}
The classical Tikhonov regularized solution for $m = Af + \varepsilon$ defined in (ref here) is usually denoted by the vector $T_\alpha(m)\in\R^n$
that minimizes

\begin{equation*}
\lnorm AT_\alpha(m) - m \rnorm^2 + \alpha \lnorm T_\alpha(m) \rnorm^2 \Leftrightarrow
\end{equation*}
\begin{equation*}
T_\alpha(m) = \underset{z\in\R^n}{\argmin}
\left\{ \lnorm Az - m \rnorm^2 + \alpha \lnorm z \rnorm^2 \right\},
\end{equation*}
where $\alpha > 0$ is called a regularization parameter. The resulting $T_\alpha(m)$ can be understood as a compromise between two conditions, namely

\begin{enumerate}[I.]
 \item $T_\alpha(m)$ should give a small residual $AT_\alpha(m) - m$.
 \item $\lnorm T_\alpha(m) \rnorm_2$ should be small.
\end{enumerate}
The $\alpha$ parameter is used in order to tune to balance between the two conditions above.

In generalized Tikhonov regularization some prior knowledge is assumed to be known. In some cases $f$ might be known to be smooth. This information can be incorporated into the regularization by choosing 

\begin{equation}
T_\alpha(m) = \underset{z\in\R^n}{\argmin}
\left\{ \lnorm Az - m \rnorm^2 + \alpha \lnorm Lz \rnorm^2 \right\},
\end{equation}  
where $L$ is a discretized differential operator. In our model proposed in [ref here] we know the glottal impulse to be zero in an interval [mera kama hit när modellen är skriven]


\subsubsection{The conjugate gradient method}\label{subsubsec:conjgrad}

%-----------------------Results-----------------------
\section{Results}\label{sec:results}


%----------------------Discussion---------------------
\section{Discussion}\label{sec:discussion}

\newpage
\begin{thebibliography}{9}
\bibitem{samu}
	Mueller, Jennifer L. \& Siltanen Samuli \emph{Linear and Nonlinear Inverse Problems with Practical Applications}.\\
	SIAM, 1:st edition, 2012

\end{thebibliography}

\end{document}



